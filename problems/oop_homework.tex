\documentclass{problem-set}

\title{OOP Problem Set}
\subheader{SLSS Computer Science Club}
\date{September 14, 2018}
\author{David Tsukernik}

\begin{document}
\maketitle

\vspace{3ex}
For each problem, remember to use proper Object-oriented programming techniques including but not limited to: inheritance, polymorphism, and encapsulation. Please do not use unnecessary if statements.

\section{Shapes}
% \difficulty{Easy}

Create a program that allows the user to choose between three shapes: a triangle, a circle, a rectangle, and a square (which is also a rectangle). The program should then ask the user for information about the shape they chose - base and height for a triangle, length and width for a rectangle, radius for a circle, and side length for a square. The program will then output the area of the shape.

\section{Vehicle Renting}
% \difficulty{Medium}

Write a vehicle renting program. This program must allow the users to rent a number of different types of vehicles including car, van, truck, and tractor trailer. Each vehicle will have common information for the user to see and choose from including make, model, seating capacity, fuel efficiency per 100km, colour, and of course, daily cost. However some vehicles will have specific information about them. Cars and Vans can be either have 2 seats or 4 seats, they can also potentially have a sunroof. Trucks will have a certain payload (weight capacity) they can carry while tractor trailers will have a payload as well as a number of axles.

Create this program and make sure it is easy to use for the customer to rent a vehicle, and when they have confirmed their selection show an invoice which describes all their choices as well as their cost (with a tax rate of 13\%).

\section{Abstraction vs Encapsulation}
% \difficulty{Hard}

How would you describe the fundamental difference between \textit{encapsulation} and \textit{abstraction} to a classroom of 7th grade students with no previous knowledge of OOP?

\end{document}