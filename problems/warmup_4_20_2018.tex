\documentclass[12pt]{article}
\usepackage{graphicx} % Allows for eps images
\usepackage[dvips,letterpaper,margin=1in,bottom=0.7in]{geometry}
\usepackage{tensor}

 % Sets margins and page size
\usepackage{xcolor}
\usepackage[framemethod=TikZ]{mdframed}
\usepackage[utf8]{inputenc}	
\usepackage[T1]{fontenc}

% \usepackage{mathpazo} % Use the Palatino font by default
\usepackage{parskip}
\usepackage{relsize}
\usepackage{xifthen}
\usepackage{fontawesome}
\usepackage{amsthm} % Theorem Formatting
\usepackage{amssymb}    % Math symbols such as \mathbb
\usepackage{amsmath}
    
    
% ************* Problem formatting helpers ****************
\newlength{\rulelength}
\newcommand{\inputformat}{\vspace{\baselineskip}\par\noindent\textbf{Input Format}\\[0.2\baselineskip]}
    
\newcommand{\constraints}{\vspace{\baselineskip}\par\noindent\textbf{Constraints}\\[0.2\baselineskip]}
\newcommand{\outputformat}{\vspace{\baselineskip}\par\noindent\textbf{Output Format}\\[0.2\baselineskip]}
    
\newcommand{\addsample}[2]{\begin{sample}{\par\noindent\textbf{Input}\\[0.2\baselineskip]\texttt{#1}}\vspace{0.5em}\\\setlength{\rulelength}{\textwidth}\addtolength{\rulelength}{2em}\hspace*{-1em}\noindent{\rule{\rulelength}{0.4pt}}{\par\noindent\textbf{Output}\\[0.2\baselineskip]\texttt{#2}}\end{sample}}
    
\newcommand{\addsampleExplanation}[3]{\begin{sample}{\par\noindent\textbf{Input}\\[0.2\baselineskip]\texttt{#1}}\vspace{0.5em}\\\setlength{\rulelength}{\textwidth}\addtolength{\rulelength}{2em}\hspace*{-1em}\noindent{\rule{\rulelength}{0.4pt}}{\par\noindent\textbf{Output}\\[0.2\baselineskip]\texttt{#2}}
\\\setlength{\rulelength}{\textwidth}\addtolength{\rulelength}{2em}\hspace*{-1em}\noindent{\rule{\rulelength}{0.4pt}}{\par\noindent\textbf{Explanation}\\[0.2\baselineskip]#3}\end{sample}}
    
\mdfdefinestyle{problemstyle}{frametitlerule=true,frametitlebackgroundcolor=gray!20,innertopmargin=\topskip,skipabove=2cm,leftmargin=1em,rightmargin=1em}
\mdtheorem[style=problemstyle]{problem}{Problem}
\mdfdefinestyle{samplestyle}{frametitlerule=true,frametitlebackgroundcolor=gray!20,innertopmargin=\topskip,innerleftmargin=1em,innerrightmargin=1em,outerlinecolor=gray!20!gray!20,outerlinewidth=0.4pt,innerbottommargin=0.8em}
\mdtheorem[style=samplestyle]{sample}{Sample}
\renewcommand\thesample{\Roman{sample}}
    
\newcommand{\notewarning}[1]{\faWarning\, \textbf{Note:} #1}
\newcommand{\nextpagesign}{\notewarning{The problem specification continues on the next page.}}
\newcommand{\pushnewpage}
{
    \vfill
    \nextpagesign
    \newpage
}

\makeatletter
\@addtoreset{sample}{section}
\makeatother
    
\newcommand{\makeHeader}
{
    {\noindent\fontseries{b}\fontsize{24}{30}\selectfont\problemsetTitle}
    \\[2\baselineskip]
    {{\bf\problemsetSubheader}\\
    {\textit{\problemsetDate}}}
    \vspace{1em}
    \hrule
}

\initSettings{Warmup Problems}{SLSS Computer Science Club}
             {April 20, 2018}{Shon Verch}

\begin{document}
\makeHeader

\section{Multiples of 3 and 5}
If we list all the natural numbers below 10 that are multiples of 3 or 5, we get 3, 5, 6 and 9. The sum of these multiples is 23.

Find the sum of all the multiples of 3 or 5 below a non-negative integer $n$.

\constraints
$1 \leq n \leq 10^{12}$

\inputformat
The first line of input describes the non-negative integer $n$.

\outputformat
Print the sum of all the multiples of 3 or 5 below a non-negative integer $n$.

\addsampleExplanation
{
10
}
{
23
}
{
3, 5, 6, and 9 are the natural numbers that are multiples of 3 or 5 below 10. The sum of these multiples is 23.
}

\newpage\section{Daily Temperatures}
Andy is recording temperatures for his meteorology class. Everyday, at exactly \texttt{12:00 PM}, he records the temperature. As part of his research, Andy wants to know, for each day, how many days you would have to wait until a warmer temperature.

Write a program to produce a list that, for each day in the input, tells you how many days you would have to wait until a warmer temperature occurs.

\constraints
$1 \leq n \leq 30000$\\
$30 \leq t_i \leq 100$

\inputformat
The first line of input contains the non-negative integer $n$ which describes the amount of days which Andy recorded data for.

The second line of input describes $n$ space-separated non-negative integers where each integer $t_i$ represents the temperature of day $i$.

\outputformat
Print $n$ space separated integers describing how many days you would have to wait until a warmer temperature occurs. If there is no future day for which this is possible, print \texttt{0}.

\addsample
{
8\\
73 74 75 71 69 72 76 73
}
{
1 1 4 2 1 1 0 0
}

\end{document}