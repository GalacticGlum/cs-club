\documentclass{problem-set}

\title{Warmup Problems}
\subheader{SLSS Computer Science Club}
\date{April 20, 2018}
\author{Shon Verch}

\begin{document}
\maketitle

\section{Multiples of 3 and 5}
If we list all the natural numbers below 10 that are multiples of 3 or 5, we get 3, 5, 6 and 9. The sum of these multiples is 23.

Find the sum of all the multiples of 3 or 5 below a non-negative integer $n$.

\constraints
$1 \leq n \leq 10^{12}$

\inputformat
The first line of input describes the non-negative integer $n$.

\outputformat
Print the sum of all the multiples of 3 or 5 below a non-negative integer $n$.

\addsampleExplanation
{
10
}
{
23
}
{
3, 5, 6, and 9 are the natural numbers that are multiples of 3 or 5 below 10. The sum of these multiples is 23.
}

\newpage\section{Daily Temperatures}
Andy is recording temperatures for his meteorology class. Everyday, at exactly \texttt{12:00 PM}, he records the temperature. As part of his research, Andy wants to know, for each day, how many days you would have to wait until a warmer temperature.

Write a program to produce a list that, for each day in the input, tells you how many days you would have to wait until a warmer temperature occurs.

\constraints
$1 \leq n \leq 30000$\\
$30 \leq t_i \leq 100$

\inputformat
The first line of input contains the non-negative integer $n$ which describes the amount of days which Andy recorded data for.

The second line of input describes $n$ space-separated non-negative integers where each integer $t_i$ represents the temperature of day $i$.

\outputformat
Print $n$ space separated integers describing how many days you would have to wait until a warmer temperature occurs. If there is no future day for which this is possible, print \texttt{0}.

\addsample
{
8\\
73 74 75 71 69 72 76 73
}
{
1 1 4 2 1 1 0 0
}

\end{document}