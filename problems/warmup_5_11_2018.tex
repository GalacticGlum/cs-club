\documentclass[12pt]{article}

\usepackage{graphicx} % Allows for eps images
\usepackage[dvips,letterpaper,margin=1in,bottom=0.7in]{geometry}
\usepackage{tensor}

 % Sets margins and page size
\usepackage{xcolor}
\usepackage[framemethod=TikZ]{mdframed}
\usepackage[utf8]{inputenc}	
\usepackage[T1]{fontenc}

% \usepackage{mathpazo} % Use the Palatino font by default
\usepackage{parskip}
\usepackage{relsize}
\usepackage{xifthen}
\usepackage{fontawesome}
\usepackage{amsthm} % Theorem Formatting
\usepackage{amssymb}    % Math symbols such as \mathbb
\usepackage{amsmath}
    
    
% ************* Problem formatting helpers ****************
\newlength{\rulelength}
\newcommand{\inputformat}{\vspace{\baselineskip}\par\noindent\textbf{Input Format}\\[0.2\baselineskip]}
    
\newcommand{\constraints}{\vspace{\baselineskip}\par\noindent\textbf{Constraints}\\[0.2\baselineskip]}
\newcommand{\outputformat}{\vspace{\baselineskip}\par\noindent\textbf{Output Format}\\[0.2\baselineskip]}
    
\newcommand{\addsample}[2]{\begin{sample}{\par\noindent\textbf{Input}\\[0.2\baselineskip]\texttt{#1}}\vspace{0.5em}\\\setlength{\rulelength}{\textwidth}\addtolength{\rulelength}{2em}\hspace*{-1em}\noindent{\rule{\rulelength}{0.4pt}}{\par\noindent\textbf{Output}\\[0.2\baselineskip]\texttt{#2}}\end{sample}}
    
\newcommand{\addsampleExplanation}[3]{\begin{sample}{\par\noindent\textbf{Input}\\[0.2\baselineskip]\texttt{#1}}\vspace{0.5em}\\\setlength{\rulelength}{\textwidth}\addtolength{\rulelength}{2em}\hspace*{-1em}\noindent{\rule{\rulelength}{0.4pt}}{\par\noindent\textbf{Output}\\[0.2\baselineskip]\texttt{#2}}
\\\setlength{\rulelength}{\textwidth}\addtolength{\rulelength}{2em}\hspace*{-1em}\noindent{\rule{\rulelength}{0.4pt}}{\par\noindent\textbf{Explanation}\\[0.2\baselineskip]#3}\end{sample}}
    
\mdfdefinestyle{problemstyle}{frametitlerule=true,frametitlebackgroundcolor=gray!20,innertopmargin=\topskip,skipabove=2cm,leftmargin=1em,rightmargin=1em}
\mdtheorem[style=problemstyle]{problem}{Problem}
\mdfdefinestyle{samplestyle}{frametitlerule=true,frametitlebackgroundcolor=gray!20,innertopmargin=\topskip,innerleftmargin=1em,innerrightmargin=1em,outerlinecolor=gray!20!gray!20,outerlinewidth=0.4pt,innerbottommargin=0.8em}
\mdtheorem[style=samplestyle]{sample}{Sample}
\renewcommand\thesample{\Roman{sample}}
    
\newcommand{\notewarning}[1]{\faWarning\, \textbf{Note:} #1}
\newcommand{\nextpagesign}{\notewarning{The problem specification continues on the next page.}}
\newcommand{\pushnewpage}
{
    \vfill
    \nextpagesign
    \newpage
}

\makeatletter
\@addtoreset{sample}{section}
\makeatother

\newcommand{\initSettings}[4]
{
        % ********************** Settings *************************
    \newcommand{\problemsetTitle}{#1}
    \newcommand{\problemsetSubheader}{#2}
    \newcommand{\problemsetDate}{#3}
    \newcommand{\problemsetAuthor}{#4}
}

\newcommand{\makeHeader}
{
    {\noindent\fontseries{b}\fontsize{24}{30}\selectfont\problemsetTitle}
    \\[2\baselineskip]
    {{\bf\problemsetSubheader}\\
    {\textit{\problemsetDate}}}
    \vspace{1em}
    \hrule
}

\initSettings{Warmup Problems}{SLSS Computer Science Club}
             {May 11, 2018}{Shon Verch}

\begin{document}
\makeHeader

\section{Garden of Triangles}
On his trip to the Italian region of Tuscany, Mr. Lane encountered an interesting Renaissance-era garden where a sacred religious ceremony used to take place. During the ceremony, there would be $n$ stones scattered around the garden. Each stone has a coordinate $(x,y)$ where $x$ can either be $1$, $2$, or $3$. The priests would then choose every triplet of these $n$ points and make a triangle from it. 

Determine the sum of areas of all the triangles that the priests can make.\\

\notewarning{Some of the triplets may not from proper triangles---or triangles at all---but instead form a line or point (i.e. degenerate). This is fine because their area will be zero.}
 
\constraints
$1 \leq n \leq 2000$\\
$1 \leq x \leq 3$\\
$1 \leq y \leq 10^6$\\
$(x_i,\;y_i) \neq (x_j,\;y_j)$

\inputformat
The first line of input describes the non-negative integer $n$ denoting the number of points in the garden. The next $n$ lines of input contain two space-seperated integers $x$ and $y$ describing the cooridnates of the $i$th point.

\outputformat
Output a single line containing the sum of areas of all the triangles that the priests can make.

\pushnewpage

\addsampleExplanation
{
3\\
1 1\\
2 1\\
3 3
}
{
1.0
}
{
There only exists one triangle which has a non-zero area of $1$.
}

\addsampleExplanation
{
4\\
1 1\\
2 2\\
2 1\\
3 3
}
{
2.0
}
{
Let the points be $A(1,1)$, $B(2,2)$, $C(2,1)$, and $D(3,3)$. 

There are three triangles possible: $ABC$, $BCD$, and $ACD$ with areas $1/2$, $1/2$, and $1$ respectively. These areas yield a total area of $2$.
}

\pushnewpage

\section*{Helpful Resources}
\begin{itemize}
    \item \textit{Desmos} is a great tool for visualising lines, curves, points, and other geometrical concepts (e.g. graphing). 
\end{itemize}

\subsection*{Distance Formula}
The distance of two points $(x_1,\;y_1)$ and $(x_2,\;y_2)$ is defined as
\begin{equation*}
    D=\sqrt{(x_1-x_2)^2+(y_1-y_2)^2}.
\end{equation*}

\subsection*{Heron's Formula}
\textit{Heron's formula} gives the area of a triangle given the side lengths. It states that the area of a triangle with side lengths $A$, $B$, and $C$ is
\begin{equation*}
    A=\sqrt{S(S-A)(S-B)(S-C)},
\end{equation*}
where $s$ is the \textit{semipermieter} of the triangle defined as
\begin{equation*}
    S=\frac{A+B+C}{2}.
\end{equation*}

Heron's formula can also be written as
\begin{equation*}
    \begin{split}
        A &= \frac{1}{4}\sqrt{(A+B+C)(-A+B+C)(A-B+C)(A+B-C)};\\
        A &= \frac{1}{4}\sqrt{2(A^2B^2+A^2C^2+B^2C^2)-(A^4+B^4+C^4)};\\
        A &= \frac{1}{4}\sqrt{(A^2+B^2+C^2)^2-2(A^4+B^4+C^4)};\\
        A &= \frac{1}{4}\sqrt{4(A^2B^2+A^2C^2+B^2C^2)-(A^2+B^2+C^2)^2}.
    \end{split}
\end{equation*}

\end{document}