\documentclass{contest-set}
\usepackage{textcomp}
\usepackage{calligra}
\DeclareMathAlphabet{\mathcalligra}{T1}{calligra}{m}{n}

\title{4M Contest}
\motto{Mock mini-mini-mini ECOO}
\subheader{SLSS Computer Science Club}
\date{October 5, 2018}
% \author{Shon Verch}

\begin{document}
\maketitle

\section{Day of Pepe}
Every year, the gods of \textit{Kekistan}---overseers of the Universe\textsuperscript{TM}---hold a competition in order to honour the founding fathers of Kekistan and the Cult of Kek. The winner of this competition is crowned Kek, the ruler of Kekistan.  This day has come to be known as the \textit{Day of Pepe} because the Kek is the mortal embodiment of \textit{Pepe the Frog}---the supreme god of Kekistan.

The competition is simple: find the pair of earthlings whose mutual distance is the smallest. The person who finds this pair wins; however, the locations of the humans may be provided in large $m$-dimensions. 

This challenge is especially difficult for Quetzalli, whose senses are restricted to only the $4$-th dimension. Given $n$ $m$-dimensional points $P_1, P_2,\ldots,P_n$, help Quetzalli find two points $\mathcal{P}$ and $\mathcal{Q}$ whose mutual distance---given by \begin{equation*}
    D(P, Q)=\sqrt{\sum^m_{i=1}(\mathcal{P}_i-\mathcal{Q}_i)^2},
\end{equation*} where $\mathcal{P}_i$ represents the $i$-th dimensional component of the point $\mathcal{P}$---is the smallest.

\inputformat
The first of input contains two space-separated integers $n$ and $m$ describing the amount of points and the dimension of the points respectively.

The next $n$ lines contain $m$ space-separated integers denoting the $i$-th dimensional component of point $p$.

\constraints
$1 \leq n \leq 10^5$\\
$1 \leq m \leq 10$\\
$1 \leq P_i \leq 10^8$

\pushnewpage

\outputformat
Output the mutual distance of the two closest points $\mathcal{P}$ and $\mathcal{Q}$, rounded to $2$ decimal places.

\addsampleExplanation
{
5 1\\
-5\\
0\\
-1\\
2\\
3
}
{
1
}
{
The points $(0)$ and $(-1)$ result in the smallest distance.
}

\addsample
{
6 2\\
2 3\\
12 30\\
40 50\\
5 1\\
12 10\\
3 4
}
{
1.41
}

\end{document}