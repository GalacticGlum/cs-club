\documentclass{cspresentation}

\title{Dynamic Programming}
\subtitle{Part \rom{2}: Advanced Applications}
\institute{Stephen Lewis Secondary School}
\author{Computer Science Club}
\date{October 23, 2019}

\makeatletter
\def\blfootnote{\gdef\@thefnmark{}\@footnotetext}
\makeatother

\DeclareMathOperator{\lortext}{\,\textrm{or}\,}

\begin{document}
\begin{frame}
\titlepage
\end{frame}

\section{Review}
\subsection{Analysing Dynamic Programs}
\begin{frame}{Five (easy) steps!}
    \begin{enumerate}
        \item \textbf{Define} the subproblemm
        \item \textbf{Guess} solutions to the subproblem
        \item \textbf{Relate} the subproblems (i.e. recurrence)
        \item \textbf{Optimize} the recursive algorithm using memoization or bottom-up
        \item \textbf{Solve} the final problem
    \end{enumerate}
\blfootnote{Source: Demaine, Erik. (2011). \textit{6.006: Introduction to Algorithms}.}
\end{frame}

\begin{frame}{Fibonacci Numbers}
    \begin{enumerate}
        \item \textbf{Define:} $F_k$ for $1 \leq k \leq n$.
        \item \textbf{Guess:} none
        \item \textbf{Relate:} $F_n=F_{n-1}+F_{n-2}$ for $n \geq 2$, where $F_1=F_2=1$.
        \item \textbf{Optimize:} memoization or bottom-up algorithm
        \item \textbf{Solve:} original problem is $F_n$.
    \end{enumerate}
    
    \textbf{Running time} is $\mathcal{O}(n)$; the \textit{number of subproblems} $(n)$ multiplied by the \textit{time per subproblem} ($\mathcal{O}(1)$ per subproblem).
\end{frame}

\begin{frame}{In general...}
    The running time of a dynamic program is
    \begin{equation*}
        \Theta\left(\beta(n)\right) + S\cdot\Theta\left(F(k)\right)
    \end{equation*}
    where $S$ is the number of subproblems, $\Theta\left(F(k)\right)$ is the time per subproblem, and $\Theta\left(\beta(n)\right)$ is the additional time to compute the final problem.
\end{frame}

\section{The Pairing Problem}
\begin{frame}{Problem Description}
    At a party of $n$ people, a person can talk with another (pair up) or remain single. Each person can only be paired once.\\~\\
    Determine the number of ways that the $n$ people can remain single or be paired up.
\end{frame}

\begin{frame}[fragile,allowframebreaks]{Analysis}
    \begin{enumerate}
        \item \textbf{Define:} the number of ways to pair $k$ people for $1 \leq k \leq n$.
        \item \textbf{Guess:} Person $k$ can either remain single or pair up.
        \item \textbf{Relate:} If $f(k)$ is the number of ways that $k$ people can be paired or single, then $$f(k)=\underbrace{f(k-1)}_{\text{single}}+\underbrace{(k-1)f(k-2)}_{\text{pair up}}$$ for $k>1$, where $f(0)=f(1)=1$.
        \framebreak
        \item \textbf{Optimize:}
\begin{lstlisting}[language=iPython,mathescape=true]
memo = {}
def dp(k):
    # Base cases
    if k < 0: return 0
    if k <= 1: return 1
    
    # Compute ${\color{ipython_cyan} f(k)}$ and memoize
    if k in memo: return memo[k]
    result = dp(k - 1) + (k - 1) * dp(k - 2)
    memo[k] = result
    return result
\end{lstlisting}
    \item \textbf{Solve:} Compute and print the value of \lstinline[language=iPython]!dp(n)!
    \end{enumerate}
\end{frame}

\section{Subset Sum}
\begin{frame}{Problem Description} 
    Given an array $A$ of integers, determine whether there exists a subset in $A$ such that the sum of the elements in the subset is equal to $S$.
\end{frame}

\begin{frame}[fragile,allowframebreaks]{Analysis}
    \begin{enumerate}
        \item \textbf{Define:} whether a subset with sum $s$ exists in array $A$.
        \item \textbf{Guess:} for all items (starting from the back): include the item in the sum \textit{or} exclude the item from the sum.
        \item \textbf{Relate:} Let $F(n,s)$ indicate whether there is a subset with sum $s$ in the subarray $A_1,\ldots,A_n$: 
        $$F(n,s)=\underbrace{F(n-1,s-A_n)}_{\text{include $A_n$}}\lortext\underbrace{F(n-1,s)}_{\text{exclude $A_n$}},$$
        where $F(n,0)$ is true and $F(0,s)$ is false.
        \framebreak
        \item \textbf{Optimize}:
        \begin{itemize}
            \item In the worst case, we have to check all possible subsets which means that the direct implementation of $F(n,s)$ runs in \textit{exponential time}. \item There are \textbf{no} known polynomial time solutions for this problem.
        \end{itemize}
        We achieve a \textit{pseudo-polynomial} time solution using dynamic programming.
        \framebreak
\begin{lstlisting}[language=iPython,mathescape=true]
# Memoized solution

memo = {}
def dp(n, s):
    if s == 0: return True
    if n == 0 and s > 0: return False
    
    if (n, s) in memo: return memo[(n, s)]
    
    if values[n - 1] > s: result = dp(n - 1, s)
    else: result = dp(n - 1, s) or dp(n - 1, s - values[n - 1])
    
    memo[(n, s)] = result
    return result
\end{lstlisting}
In this case, the memoized solution does \textit{not} have the same running time as a bottom-up approach.\\~\\
\textbf{Bottom-up approach:}
\begin{itemize}
    \item Let $\mathcal{F}(i,j)$ denote whether there exists a subset from $\{A_1,\ldots,A_j\}$ with sum $i$.
    \item We compute $\mathcal{F}(i,j)$ for all $1 \leq i \leq |A|$ and $1 \leq j \leq S$.
\end{itemize}
\item \textbf{Solve}: Print the value of $\mathcal{F}\left(S,|A|\right)$.
    \end{enumerate}
\end{frame}

\end{document}