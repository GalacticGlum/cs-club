\documentclass{beamer}
\usepackage[utf8]{inputenc}
\usepackage{tikz}
\usepackage[framemethod=tikz]{mdframed}
\usepackage{mathtools}
\usepackage{breqn}
\usetikzlibrary{positioning,shapes,fit,arrows}
\graphicspath{ {./images/} }

\usepackage{xcolor}
\definecolor{maroon}{cmyk}{0, 0.87, 0.68, 0.32}
\definecolor{halfgray}{gray}{0.55}
\definecolor{ipython_frame}{RGB}{207, 207, 207}
\definecolor{ipython_bg}{RGB}{247, 247, 247}
\definecolor{ipython_red}{RGB}{186, 33, 33}
\definecolor{ipython_green}{RGB}{0, 128, 0}
\definecolor{ipython_cyan}{RGB}{64, 128, 128}
\definecolor{ipython_purple}{RGB}{170, 34, 255}

\usepackage{listings}
\lstset{
    breaklines=true,
    %
    extendedchars=true,
    literate=
    {á}{{\'a}}1 {é}{{\'e}}1 {í}{{\'i}}1 {ó}{{\'o}}1 {ú}{{\'u}}1
    {Á}{{\'A}}1 {É}{{\'E}}1 {Í}{{\'I}}1 {Ó}{{\'O}}1 {Ú}{{\'U}}1
    {à}{{\`a}}1 {è}{{\`e}}1 {ì}{{\`i}}1 {ò}{{\`o}}1 {ù}{{\`u}}1
    {À}{{\`A}}1 {È}{{\'E}}1 {Ì}{{\`I}}1 {Ò}{{\`O}}1 {Ù}{{\`U}}1
    {ä}{{\"a}}1 {ë}{{\"e}}1 {ï}{{\"i}}1 {ö}{{\"o}}1 {ü}{{\"u}}1
    {Ä}{{\"A}}1 {Ë}{{\"E}}1 {Ï}{{\"I}}1 {Ö}{{\"O}}1 {Ü}{{\"U}}1
    {â}{{\^a}}1 {ê}{{\^e}}1 {î}{{\^i}}1 {ô}{{\^o}}1 {û}{{\^u}}1
    {Â}{{\^A}}1 {Ê}{{\^E}}1 {Î}{{\^I}}1 {Ô}{{\^O}}1 {Û}{{\^U}}1
    {œ}{{\oe}}1 {Œ}{{\OE}}1 {æ}{{\ae}}1 {Æ}{{\AE}}1 {ß}{{\ss}}1
    {ç}{{\c c}}1 {Ç}{{\c C}}1 {ø}{{\o}}1 {å}{{\r a}}1 {Å}{{\r A}}1
    {€}{{\EUR}}1 {£}{{\pounds}}1
}

\usepackage{animate}

%%
%% Python definition (c) 1998 Michael Weber
%% Additional definitions (2013) Alexis Dimitriadis
%% modified by me (should not have empty lines)
%%
\lstdefinelanguage{iPython}{
    morekeywords={access,and,break,class,continue,def,del,elif,else,except,exec,finally,for,from,global,if,import,in,is,lambda,not,or,pass,print,raise,return,try,while},%
    %
    % Built-ins
    morekeywords=[2]{abs,all,any,basestring,bin,bool,bytearray,callable,chr,classmethod,cmp,compile,complex,delattr,dict,dir,divmod,enumerate,eval,execfile,file,filter,float,format,frozenset,getattr,globals,hasattr,hash,help,hex,id,input,int,isinstance,issubclass,iter,len,list,locals,long,map,max,memoryview,min,next,object,oct,open,ord,pow,property,range,raw_input,reduce,reload,repr,reversed,round,set,setattr,slice,sorted,staticmethod,str,sum,super,tuple,type,unichr,unicode,vars,xrange,zip,apply,buffer,coerce,intern},%
    %
    sensitive=true,%
    morecomment=[l]\#,%
    morestring=[b]',%
    morestring=[b]",%
    %
    morestring=[s]{'''}{'''},% used for documentation text (mulitiline strings)
    morestring=[s]{"""}{"""},% added by Philipp Matthias Hahn
    %
    morestring=[s]{r'}{'},% `raw' strings
    morestring=[s]{r"}{"},%
    morestring=[s]{r'''}{'''},%
    morestring=[s]{r"""}{"""},%
    morestring=[s]{u'}{'},% unicode strings
    morestring=[s]{u"}{"},%
    morestring=[s]{u'''}{'''},%
    morestring=[s]{u"""}{"""},%
    %
    % {replace}{replacement}{lenght of replace}
    % *{-}{-}{1} will not replace in comments and so on
    literate=
    {á}{{\'a}}1 {é}{{\'e}}1 {í}{{\'i}}1 {ó}{{\'o}}1 {ú}{{\'u}}1
    {Á}{{\'A}}1 {É}{{\'E}}1 {Í}{{\'I}}1 {Ó}{{\'O}}1 {Ú}{{\'U}}1
    {à}{{\`a}}1 {è}{{\`e}}1 {ì}{{\`i}}1 {ò}{{\`o}}1 {ù}{{\`u}}1
    {À}{{\`A}}1 {È}{{\'E}}1 {Ì}{{\`I}}1 {Ò}{{\`O}}1 {Ù}{{\`U}}1
    {ä}{{\"a}}1 {ë}{{\"e}}1 {ï}{{\"i}}1 {ö}{{\"o}}1 {ü}{{\"u}}1
    {Ä}{{\"A}}1 {Ë}{{\"E}}1 {Ï}{{\"I}}1 {Ö}{{\"O}}1 {Ü}{{\"U}}1
    {â}{{\^a}}1 {ê}{{\^e}}1 {î}{{\^i}}1 {ô}{{\^o}}1 {û}{{\^u}}1
    {Â}{{\^A}}1 {Ê}{{\^E}}1 {Î}{{\^I}}1 {Ô}{{\^O}}1 {Û}{{\^U}}1
    {œ}{{\oe}}1 {Œ}{{\OE}}1 {æ}{{\ae}}1 {Æ}{{\AE}}1 {ß}{{\ss}}1
    {ç}{{\c c}}1 {Ç}{{\c C}}1 {ø}{{\o}}1 {å}{{\r a}}1 {Å}{{\r A}}1
    {€}{{\EUR}}1 {£}{{\pounds}}1
    %
    {^}{{{\color{ipython_purple}\^{}}}}1
    {=}{{{\color{ipython_purple}=}}}1
    %
    {+}{{{\color{ipython_purple}+}}}1
    {*}{{{\color{ipython_purple}$^\ast$}}}1
    {/}{{{\color{ipython_purple}/}}}1
    %
    {+=}{{{+=}}}1
    {-=}{{{-=}}}1
    {*=}{{{$^\ast$=}}}1
    {/=}{{{/=}}}1,
    literate=
    *{-}{{{\color{ipython_purple}-}}}1
     {?}{{{\color{ipython_purple}?}}}1,
    %
    identifierstyle=\color{black}\ttfamily,
    commentstyle=\color{ipython_cyan}\ttfamily,
    stringstyle=\color{ipython_red}\ttfamily,
    keepspaces=true,
    showspaces=false,
    showstringspaces=false,
    %
    rulecolor=\color{ipython_frame},
    frame=single,
    frameround={t}{t}{t}{t},
    framexleftmargin=6mm,
    numbers=left,
    numberstyle=\tiny\color{halfgray},
    %
    %
    backgroundcolor=\color{ipython_bg},
    %   extendedchars=true,
    basicstyle=\scriptsize,
    keywordstyle=\color{ipython_green}\ttfamily,
}

% \definecolor{light-gray}{gray}{0.95}
\surroundwithmdframed[
  hidealllines=true,
  innerleftmargin=15pt,
  innertopmargin=0pt,
  innerbottommargin=0pt]{lstlisting}

\usetheme{Berlin}
\usefonttheme[onlymath]{serif}

% \usecolortheme{crane}
\setbeamertemplate{navigation symbols}{}%remove navigation symbols

\setbeamertemplate{frametitle continuation}[from second][(contd.)]

\newcommand{\sectionFrame}[3]
{
    \section{#3}
    \begin{frame}
    \begin{block}{}
    \begin{center}
        \Huge{#1}\\[0.5ex]
        \large{#2}
    \end{center}
    \end{block}
    \end{frame}
}

\usepackage{caption}
\DeclareMathOperator{\MCE}{MCE}

\makeatletter
\newcommand*{\rom}[1]{\expandafter\@slowromancap\romannumeral #1@}
\makeatother

\title{Dynamic Programming}
\subtitle{Part \rom{1}: Introduction}
\institute{Stephen Lewis Secondary School}
\author{Computer Science Club}
\date{October 15, 2019}

\begin{document}
\begin{frame}
\titlepage
\end{frame}

\section{Introduction}
\subsection{Definition}
\begin{frame}{What is it?}
    A powerful, general-purpose algorithmic design technique. It is especially good at, and is generally used for optimization problems.
    \begin{itemize}
        \item Dynamic programming is an optimized brute-force
    \end{itemize}
\end{frame}

\begin{frame}{When is it used?}
    Dynamic programming is really good when we can break down our problem into smaller subproblems of the same nature.\\~\\We \textit{reuse} previous solutions to efficiently find new solutions.
\end{frame}

\section{Fibonacci}
\subsection{Example Problem (Fibonacci)}
\begin{frame}{Problem Description}
If:
    \begin{itemize}
        \item $F_1=F_2=1$
        \item $F_n=F_{n-1}+F_{n-2}$ where $n > 2$
    \end{itemize}
    
then, compute for $F_n$.
\end{frame}

\begin{frame}[fragile]{Na\"{i}ve Solution}
    We can implement the algorithm directly as a recursive function:
\begin{center}
\begin{lstlisting}[language=iPython]
def fib(n):
    if n <= 2: return 1
    return fib(n - 1) + fib(n-2)
\end{lstlisting}
\end{center}

\textbf{Running Time}: Exponential
\end{frame}

\subsection{Memoization}
\begin{frame}[fragile]{DP Solution \rom{1} (Memoization)}
    \begin{itemize}
        \item There is overlap between computing successive Fibonacci numbers (i.e. computing $F_n$ and $F_{n-1}$ both require $F_{n-2}$).
        \item \textbf{Memoization}: only compute once.
    \end{itemize}

\begin{center}
\begin{lstlisting}[language=iPython, mathescape=true]
memo = {}
def fib(n):
    # Check if we already computed ${\color{ipython_cyan} F_n}$
    if n in memo: return memo[n]
    
    # Compute ${\color{ipython_cyan} F_n}$
    if n <= 2: result = 1
    else: result = fib(n - 1) + fib(n-2)
    
    memo[n] = result
    return result
\end{lstlisting}
\end{center}
\end{frame}

\begin{frame}{In general...}
    We \textit{memoize} solutions to \textit{subproblems} so that we can reuse them to solve the actual problem.\\~\\
    A subproblem is a \textbf{smaller} version of the same problem.
    \begin{block}{Example: Fibonacci}
        \begin{itemize}
            \item Our subproblems are computing $F_{n-1},F_{n-2},\ldots,F_1$.
            \item Our actual problem is computing $F_n$.
        \end{itemize}
    \end{block}
\end{frame}

\subsection{Bottom-up algorithm}
\begin{frame}{Introduction}
    \begin{itemize}
        \item A memoized recursive algorithm thinks of the problem starting from the top and works its way down (the final problem we want to solve).
        \item A bottom-up algorithm does the reverse: starts from the bottom and works its way up to the final problem.
    \end{itemize}
\end{frame}

\begin{frame}[fragile]{DP Solution \rom{2} (Bottom-up)}
    \textbf{Strategy:} Starting from $k=1$, we successively compute the next term, $F_{k+1}$, until we reach $F_n$.
\begin{center}
\begin{lstlisting}[language=iPython, mathescape=true]
def fibonacci(n):
    fib = {}
    for k in range(1, n + 1):
        if k <= 2: result = 1
        else: result = fib[k - 1] + fib[k - 2]
        fib[k] = result
    
    return fib[n]
\end{lstlisting}
\end{center}
\end{frame}

\begin{frame}{In general...}
    \begin{itemize}
        \item Bottom-up algorithms are essentially \textit{unravelling} the recursive memoized solution and writing it out \textit{sequentially}.
        \item Exact same running time
    \end{itemize}
\end{frame}

\section{Binomial Coefficients}
\subsection{Example Problem (Binomial Coefficients)}
\begin{frame}{Problem Description}
    \begin{itemize}
        \item A binomial coefficient, $C(n,k)$, gives the coefficient of $x^k$ in the expansion of $(1+x)^n$.
        \item It can be defined recursively as $C(n,k)=C(n-1,k-1)+C(n-1,k)$.
    \end{itemize}
    
    Compute the value of $C(n,k)$.
\end{frame}

\begin{frame}[fragile]{Solution}
    \textbf{Remember:} DP is used to \textit{optimize} problems that depend on smaller subproblems (i.e. recursion).
    Since we are given the recursive formula, we can easily apply memoization or bottom-up to get an efficient solution:
\begin{lstlisting}[language=iPython, mathescape=true]
memo = {}
def nck(n, k):
    if k > n: return 0
    
    if (n, k) in memo: return memo[(n, k)]
   
    if n == k or k == 0: result = 1
    else: result = nck(n - 1, k) + nck(n - 1, k - 1)
    
    memo[(n, k)] = result
    return result
\end{lstlisting}
\end{frame}
\end{document}