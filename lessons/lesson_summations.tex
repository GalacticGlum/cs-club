\documentclass[prereq]{cslesson}
\usepackage{amsmath}

\numberwithin{equation}{section}

\title{A Quick Overview of Summations}
\author{SLSS Computer Science Club}
\date{\today}
    
\begin{document}
    
\maketitle

As we analyse algorithms and describe their properties, we will need to express large sums over many variables. Typically, we express these sorts of sums as an 
expression in the general form
\begin{equation*}
    S=a_1+a_2+\ldots+a_n
    \label{eq:verbose_sum}
\end{equation*}
where each number $\{a_k \mid 1 \leq k \leq n\}$ has been defined. We use the `\ldots' (ellipsis) to denote that the sum should be
completed based on the pattern established by the surrounding terms. We will refer to this notation of writing sums
as \textit{``expanded sum notation.''}
 
While this notation is mathematically correct, it is rather verbose and is oftentimes ambiguous.
Sums like $1+2+\sqrt{2}+\ldots+e$ are meaningless without context because there is no pattern to the terms; that is,
it is impossible for us to complete the sum because the present terms give no way to deduce a pattern.

Often, we will also want to write sums of large expressions. Using the expanded notation can yield very large and
ugly expressions. Consider the sum of the expression $x^2+x$ where $1 \leq x \leq 5$:
\begin{equation*}
    S=(1^2+1)+(2^2+2)+(3^2+3)+(4^2+4)+(5^2+5).
\end{equation*}

Obviously, this is undesirable in real mathematics because it is much too verbose and space-consuming.
Instead, we need a terser shorthand notation for expressing sums. The \textit{summation} notation provides a convenient and powerful tool for 
expressing and manipulating sums. This lesson develops the notation and introduces general techniques, formulas, and properties.

\section{Notation}
We use the Greek letter $\sum$ to represent summation in a clean and concise manner. This notation is appropriately named \textit{Sigma-notation} or \textit{Summation notation}. These terms can be used interchangeably but for consistency, we will refer to the notation as summation-notation. We can rewrite the expanded sum $a_1+a_2+\ldots+a_n$, in summation-notation as
\begin{equation}
    \sum^n_{i=1}a_i.
    \label{eq:first_summation}
\end{equation}
This notation tells us to include in the sum the terms $a_i$ whose index $i$ lies between the lower and upper bounds $1$ and $n$ respectively; that is, we sum over the terms $a_i$, where $i$ is from $1$ to $n$.

The expression which belongs within the sum is called the \textit{summand}. The summand is the expression which we sum
over given an index and upper bound. Note that the variable that represents the index (commonly called $i$) can in actuality, be anything. We can just as easily rewrite equation \ref{eq:first_summation} using the index variable $k$:
\begin{equation}
    \sum^n_{k=1}a_k.
\end{equation}

We are also able to represent multiple constraints on our index using summation-notation. This means that we are not restricted
to consecutive integers. As an example, we can express the sum of the first $n$ odd positive integer squares as
\begin{equation}
    \sum^n_{\substack{i=1 \\ i\text{ odd}}}i^2,
\end{equation}
instead of,
\begin{equation*}
    \sum^{\floor*{n/2}}_{i=0}(2i+1)^2
\end{equation*}
which is far more complex, less clear, and cumbersome to read. 
\end{document}