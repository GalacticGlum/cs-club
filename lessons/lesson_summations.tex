\documentclass[prereq]{cslesson}
\usepackage{amsmath}

\title{A Quick Overview of Summations}
\author{SLSS Computer Science Club}
\date{\today}
    
\begin{document}
    
\maketitle

As we analyse algorithms and describe their properties, we will need to express large sums over many variables. Typically, we express these sorts of sums as an 
expression in the general form
\begin{equation*}
    S=a_1+a_2+\ldots+a_n
    \label{eq:verbose_sum}
\end{equation*}
where each number $a_k,\;1 \leq k \leq n$ has been defined. While this notation is mathematically ``fine,'' it is rather verbose and long-winded---especially 
when you are summing over a large amount of variables and expressions. Consider the sum of the expression $x^2+x$ where $1 \leq x \leq 5$:
\begin{equation}
    S=(1^2+1)+(2^2+2)+(3^2+3)+(4^2+4)+(5^2+5).
    \label{eq:verbose_sum_expression}
\end{equation}
Obviously, equation \ref{eq:verbose_sum_expression} is undesirable in real mathematics because it is much too verbose and space-consuming.
Instead, we need a terser shorthand notation for expressing sums. The \textit{summation} notation provides a convenient and powerful tool for 
expressing and manipulating sums. This lesson develops the notation and introduces general techniques, formulas, and properties.

\section{Notation}

\end{document}