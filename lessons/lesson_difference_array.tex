\documentclass{beamer}
\usepackage[utf8]{inputenc}
\usepackage{listings}
\usepackage{tikz}
\usetikzlibrary{positioning,shapes,fit,arrows}
\graphicspath{ {./images/} }

\lstset
{
    language=python,
    breaklines=true,
    basicstyle=\tt\scriptsize,
    keywordstyle=\color{blue},
    identifierstyle=\color{magenta},
}

\usetheme{Antibes}

\usecolortheme{beaver}
\usefonttheme[onlymath]{serif}

\setbeamertemplate{navigation symbols}{}

\title[Difference Arrays---Computer Science Club]{\textbf{Difference Arrays}}
\institute{Stephen Lewis Secondary School \\[3ex] {\large Computer Science Club}}
\date{April 26, 2019}

\newcommand{\sectionFrame}[3]
{
    \section{#3}
    \begin{frame}
    \begin{block}{}
    \begin{center}
        \Huge{#1}\\[0.5ex]
        \large{#2}
    \end{center}
    \end{block}
    \end{frame}
}

\begin{document}

\begin{frame} 
\titlepage 
\end{frame} 

\section{Introduction}

\begin{frame}{Problem}
\begin{itemize}
    \item You are given an empty array of size $n$ followed by $m$ updates.
    \item An update consists of integers $i$, $j$ and $k$ ($i \leq j$), for which you are to increase every element from index $i$ to $j$, inclusive, by $k$.
    \item At the end of the $m$ updates, you are to output the number of elements in the array with a value of exactly $x$.
    \item What is the most optimal solution to this problem?
\end{itemize}
\end{frame}

\begin{frame}{Example}
    \begin{itemize}
        \item $n = 5, m = 3, x = 6$
        
        \item Initial Array:
        \begin{itemize}
            \item $[0, 0, 0, 0, 0]$
        \end{itemize}
        
        \item Updates:
        \begin{itemize}
            \item Update 1: $i = 2, j = 3, k = 5$
            \begin{itemize}
                \item $[0, 0, 0 + 5, 0 + 5, 0] = [0, 0, 5, 5, 0]$
            \end{itemize}
            
            \item Update 2: $i = 0, j = 4, k = 1$
            \begin{itemize}
                \item $[0 + 1, 0 + 1, 5 + 1, 5 + 1, 0 + 1] = [1, 1, 6, 6, 1]$
            \end{itemize}
        
            \item Update 3: $i = 3, j = 3, k = -6$
            \begin{itemize}
                \item $[1, 1, 6 - 6, 6, 1] = [1, 1, 0, 6, 1]$
            \end{itemize}
        \end{itemize}
        
        \item Solution:
        \begin{itemize}
            \item $1$ element in the array have a value of $x = 6$
        \end{itemize}
    \end{itemize}
\end{frame}

\section{Intuitive Solution}
\begin{frame}[fragile]{Intuitive Solution}
    \begin{itemize}
        \item For each update, loop through the array from index $i$ to $j$, inclusive, increasing each element by $k$
    \end{itemize}

    \begin{lstlisting}
        def update(i, j, k):
            for index in range(i, j + 1):
                array[index] += k
    \end{lstlisting}
    
    \begin{itemize}
        \item At the end, loop through the array and count the number of elements of a value of exactly $x$
    \end{itemize}
    
    \begin{lstlisting}
        def count():
        
            count = 0
            for i in range(n):
                if array[i] == x:
                    count += 1
            
            return count
    \end{lstlisting}
\end{frame}

\begin{frame}{Intuitive Solution - Time Complexity}
\begin{itemize}
    \item One Query $\mathcal{O}({n)}$
    \item All Queries: $\mathcal{O}{(mn)}$
    \item Count: $\mathcal{O}{(n)}$
    \item Overall: $\mathcal{O}{(mn)}$
\end{itemize}
\end{frame}

\section{Difference Array}
\begin{frame}[fragile]{Difference Array}
    \begin{itemize}
        \item The difference array, \texttt{d}, of an array, \texttt{a}, is defined as \\ \texttt{d[i] = a[i] - a[i - 1]}
        \item As such, elements in the array can be defined as \\ \texttt{a[i] = a[i - 1] + d[i]}
        
        \item For each update, the corresponding difference array can be updated via \texttt{d[i] += k} and \texttt{d[j + 1] -= k}
    \end{itemize}
    
    \begin{lstlisting}
        def update(i, j, k):
            d[i] += k
            d[j + 1] -= k
    \end{lstlisting}
    
    \begin{itemize}
        \item One Query: $\mathcal{O}{(1)}$
        \item All Queries: $\mathcal{O}{(m)}$
    \end{itemize}
\end{frame}

\begin{frame}[fragile]{Difference Array}
    \begin{itemize}
        \item At the end of the $m$ updates, the array can be constructed via \texttt{a[i] = a[i - 1] + d[i]}
        \item The base case is defined as \texttt{a[0] = d[0]}
    \end{itemize}

    \begin{lstlisting}
        def count():
            a[0] = d[0]
            
            for i in range(1, n):
                a[i] = a[i - 1] + d[i]
            
            count = 0
            for i in range(n);
                if a[i] == x:
                    count += 1
            
            return count
    \end{lstlisting}
    
    \begin{itemize}
        \item Count: $\mathcal{O}{(n)}$
        \item Overall: $\mathcal{O}{(n + m)}$
    \end{itemize}
\end{frame}

\section{2D Difference Array}
\begin{frame}[fragile]{2D Difference Array}
    \begin{itemize}
        \item You are given an empty 2D of size $n$ by $m$ followed by $q$ updates.
        \item An update consists of $x_1, y_1, x_2, y_2, k$ ($x_1 \leq x_2$ and $y_1 \leq y_2$), for which you are to increase every element contained in the rectangle ($P_1(x_1, y_1), P_2(x_2, y_2)$) by $k$.
        \item At the end of the $q$ updates, you are to output the number of elements in the 2D array with a value of exactly $x$
    \end{itemize}
\end{frame}

\begin{frame}[fragile]{2D Difference Array}
\begin{itemize}
    \item The 2D difference array, \texttt{d}, of a 2D array, \texttt{a}, is defined as \texttt{d[i][j] = a[i][j] - a[i - 1][j] - a[i][j - 1] + a[i - 1][j - 1]}
    \item As such, elements in the 2D array can be defined as \\ \texttt{a[i][j] = d[i][j] + a[i - 1][j] + a[i][j - 1] - a[i - 1][j - 1]}
    \item For each update, the corresponding 2D difference array can be updated as follows
\end{itemize}

\begin{lstlisting}
    def update(x1, y1, x2, y2, k):
        d[x1][y1] += k
        d[x2 + 1][y1] -= k
        d[x1][y2 + 1] -= k
        d[x2 + 1][y2 + 1] += k
\end{lstlisting}
\end{frame}

\begin{frame}[fragile]{2D Difference Array Implementation}
\begin{lstlisting}
    def update(x1, y1, x2, y2, k):
        d[x1][y1] += k
        d[x2 + 1][y1] -= k
        d[x1][y2 + 1] -= k
        d[x2 + 1][y2 + 1] += k
    
    def count():
    
        count = 0
        for i in range(n):
            for j in range(m):
                a[i][j] = d[i][j]
                
                if i > 0: a[i][j] += a[i - 1][j]

                if j > 0: a[i][j] += a[i][j - 1]
                
                if i > 0 and j > 0: a[i][j] -= a[i - 1][j - 1]
        
        return count
\end{lstlisting}

\end{frame}

\section{Practice}
\begin{frame}{Practice Problems}
\begin{itemize}
    \item Battle Positions \\ \url{https://dmoj.ca/problem/seed3}
    \item SLPR 2 P5 - Class Balancing \url{https://ssoj.ca/problem/slpr2p5}
    \item DMPG '15 S5 - Black and White  \url{https://dmoj.ca/problem/dmpg15s5}
    \item USACO 2019 February Contest, Silver P2. Painting the Barn \url{http://www.usaco.org/index.php?page=viewproblem2&cpid=919}
\end{itemize}


\end{frame}

\end{document}