\documentclass{beamer}
\usepackage[utf8]{inputenc}
\usepackage{tikz}
\usepackage[framemethod=tikz]{mdframed}
\usepackage{mathtools}
\usepackage{breqn}
\usetikzlibrary{positioning,shapes,fit,arrows}
\graphicspath{ {./images/} }

\usepackage{xcolor}
\definecolor{maroon}{cmyk}{0, 0.87, 0.68, 0.32}
\definecolor{halfgray}{gray}{0.55}
\definecolor{ipython_frame}{RGB}{207, 207, 207}
\definecolor{ipython_bg}{RGB}{247, 247, 247}
\definecolor{ipython_red}{RGB}{186, 33, 33}
\definecolor{ipython_green}{RGB}{0, 128, 0}
\definecolor{ipython_cyan}{RGB}{64, 128, 128}
\definecolor{ipython_purple}{RGB}{170, 34, 255}

\usepackage{listings}
\lstset{
    breaklines=true,
    %
    extendedchars=true,
    literate=
    {á}{{\'a}}1 {é}{{\'e}}1 {í}{{\'i}}1 {ó}{{\'o}}1 {ú}{{\'u}}1
    {Á}{{\'A}}1 {É}{{\'E}}1 {Í}{{\'I}}1 {Ó}{{\'O}}1 {Ú}{{\'U}}1
    {à}{{\`a}}1 {è}{{\`e}}1 {ì}{{\`i}}1 {ò}{{\`o}}1 {ù}{{\`u}}1
    {À}{{\`A}}1 {È}{{\'E}}1 {Ì}{{\`I}}1 {Ò}{{\`O}}1 {Ù}{{\`U}}1
    {ä}{{\"a}}1 {ë}{{\"e}}1 {ï}{{\"i}}1 {ö}{{\"o}}1 {ü}{{\"u}}1
    {Ä}{{\"A}}1 {Ë}{{\"E}}1 {Ï}{{\"I}}1 {Ö}{{\"O}}1 {Ü}{{\"U}}1
    {â}{{\^a}}1 {ê}{{\^e}}1 {î}{{\^i}}1 {ô}{{\^o}}1 {û}{{\^u}}1
    {Â}{{\^A}}1 {Ê}{{\^E}}1 {Î}{{\^I}}1 {Ô}{{\^O}}1 {Û}{{\^U}}1
    {œ}{{\oe}}1 {Œ}{{\OE}}1 {æ}{{\ae}}1 {Æ}{{\AE}}1 {ß}{{\ss}}1
    {ç}{{\c c}}1 {Ç}{{\c C}}1 {ø}{{\o}}1 {å}{{\r a}}1 {Å}{{\r A}}1
    {€}{{\EUR}}1 {£}{{\pounds}}1
}

\usepackage{animate}

%%
%% Python definition (c) 1998 Michael Weber
%% Additional definitions (2013) Alexis Dimitriadis
%% modified by me (should not have empty lines)
%%
\lstdefinelanguage{iPython}{
    morekeywords={access,and,break,class,continue,def,del,elif,else,except,exec,finally,for,from,global,if,import,in,is,lambda,not,or,pass,print,raise,return,try,while},%
    %
    % Built-ins
    morekeywords=[2]{abs,all,any,basestring,bin,bool,bytearray,callable,chr,classmethod,cmp,compile,complex,delattr,dict,dir,divmod,enumerate,eval,execfile,file,filter,float,format,frozenset,getattr,globals,hasattr,hash,help,hex,id,input,int,isinstance,issubclass,iter,len,list,locals,long,map,max,memoryview,min,next,object,oct,open,ord,pow,property,range,raw_input,reduce,reload,repr,reversed,round,set,setattr,slice,sorted,staticmethod,str,sum,super,tuple,type,unichr,unicode,vars,xrange,zip,apply,buffer,coerce,intern},%
    %
    sensitive=true,%
    morecomment=[l]\#,%
    morestring=[b]',%
    morestring=[b]",%
    %
    morestring=[s]{'''}{'''},% used for documentation text (mulitiline strings)
    morestring=[s]{"""}{"""},% added by Philipp Matthias Hahn
    %
    morestring=[s]{r'}{'},% `raw' strings
    morestring=[s]{r"}{"},%
    morestring=[s]{r'''}{'''},%
    morestring=[s]{r"""}{"""},%
    morestring=[s]{u'}{'},% unicode strings
    morestring=[s]{u"}{"},%
    morestring=[s]{u'''}{'''},%
    morestring=[s]{u"""}{"""},%
    %
    % {replace}{replacement}{lenght of replace}
    % *{-}{-}{1} will not replace in comments and so on
    literate=
    {á}{{\'a}}1 {é}{{\'e}}1 {í}{{\'i}}1 {ó}{{\'o}}1 {ú}{{\'u}}1
    {Á}{{\'A}}1 {É}{{\'E}}1 {Í}{{\'I}}1 {Ó}{{\'O}}1 {Ú}{{\'U}}1
    {à}{{\`a}}1 {è}{{\`e}}1 {ì}{{\`i}}1 {ò}{{\`o}}1 {ù}{{\`u}}1
    {À}{{\`A}}1 {È}{{\'E}}1 {Ì}{{\`I}}1 {Ò}{{\`O}}1 {Ù}{{\`U}}1
    {ä}{{\"a}}1 {ë}{{\"e}}1 {ï}{{\"i}}1 {ö}{{\"o}}1 {ü}{{\"u}}1
    {Ä}{{\"A}}1 {Ë}{{\"E}}1 {Ï}{{\"I}}1 {Ö}{{\"O}}1 {Ü}{{\"U}}1
    {â}{{\^a}}1 {ê}{{\^e}}1 {î}{{\^i}}1 {ô}{{\^o}}1 {û}{{\^u}}1
    {Â}{{\^A}}1 {Ê}{{\^E}}1 {Î}{{\^I}}1 {Ô}{{\^O}}1 {Û}{{\^U}}1
    {œ}{{\oe}}1 {Œ}{{\OE}}1 {æ}{{\ae}}1 {Æ}{{\AE}}1 {ß}{{\ss}}1
    {ç}{{\c c}}1 {Ç}{{\c C}}1 {ø}{{\o}}1 {å}{{\r a}}1 {Å}{{\r A}}1
    {€}{{\EUR}}1 {£}{{\pounds}}1
    %
    {^}{{{\color{ipython_purple}\^{}}}}1
    {=}{{{\color{ipython_purple}=}}}1
    %
    {+}{{{\color{ipython_purple}+}}}1
    {*}{{{\color{ipython_purple}$^\ast$}}}1
    {/}{{{\color{ipython_purple}/}}}1
    %
    {+=}{{{+=}}}1
    {-=}{{{-=}}}1
    {*=}{{{$^\ast$=}}}1
    {/=}{{{/=}}}1,
    literate=
    *{-}{{{\color{ipython_purple}-}}}1
     {?}{{{\color{ipython_purple}?}}}1,
    %
    identifierstyle=\color{black}\ttfamily,
    commentstyle=\color{ipython_cyan}\ttfamily,
    stringstyle=\color{ipython_red}\ttfamily,
    keepspaces=true,
    showspaces=false,
    showstringspaces=false,
    %
    rulecolor=\color{ipython_frame},
    frame=single,
    frameround={t}{t}{t}{t},
    framexleftmargin=6mm,
    numbers=left,
    numberstyle=\tiny\color{halfgray},
    %
    %
    backgroundcolor=\color{ipython_bg},
    %   extendedchars=true,
    basicstyle=\scriptsize,
    keywordstyle=\color{ipython_green}\ttfamily,
}

% \definecolor{light-gray}{gray}{0.95}
\surroundwithmdframed[
  hidealllines=true,
  innerleftmargin=15pt,
  innertopmargin=0pt,
  innerbottommargin=0pt]{lstlisting}

\usetheme{Berlin}
\usefonttheme[onlymath]{serif}

% \usecolortheme{crane}
\setbeamertemplate{navigation symbols}{}%remove navigation symbols

\title{Back to School Contest '19}
\subtitle{Problem Solutions}
\institute{Stephen Lewis Secondary School}
\author{Computer Science Club}
\date{September 25, 2019}

\setbeamertemplate{frametitle continuation}[from second][(contd.)]

\newcommand{\sectionFrame}[3]
{
    \section{#3}
    \begin{frame}
    \begin{block}{}
    \begin{center}
        \Huge{#1}\\[0.5ex]
        \large{#2}
    \end{center}
    \end{block}
    \end{frame}
}

\usepackage{caption}
\DeclareMathOperator{\MCE}{MCE}

\makeatletter
\newcommand*{\rom}[1]{\expandafter\@slowromancap\romannumeral #1@}
\makeatother

\begin{document}
\begin{frame} 
\titlepage 
\end{frame}

\section{``A Coin Game''}
\subsection{Overview}
\begin{frame}{Problem Description}
    \only<1-2>{Jack and Jill are playing a game with $N$ stacks of coins, where the $i$-th stack contains $C_i$ coins ($1 \leq i \leq N$).
    \begin{itemize}
        \item The pair take turns eliminating a stack of their choice until all remaining stacks have the same number of coins.
        \item At this point, since the next player cannot go, the current player is declared the winner.
    \end{itemize}}
    
    \only<2>{If both Jack and Jill play \textit{optimally}, Jill wants to know whether she should go first or second in order to win the game.}
\end{frame}

\begin{frame}{Optimality}
    Playing optimally refers to the fact that the players will take the minimum number of moves required \textit{regardless} of whether it is advantageous to the current player.
\end{frame}

\begin{frame}[fragile]{Example}
    \only<1-2>{Suppose there are four stacks of coins: $7$, $3$, $2$, and $2$. Should Jill go first or second, and how many turns will it take for Jill to win?}
    \only<2>{\begin{block}{Solution}
        Jill should go second since Jack will remove stack 1 or 2 and then Jill will remove the remaining stack leaving the two stacks with 2 coins each.\\~\\This will take a total of $2$ turns.
    \end{block}}
\end{frame}

\subsection{Determining the length of the game}
\begin{frame}[fragile]{Prelude}
    Recall that since Jack and Jill play optimally, the game will last for the \textit{minimum number of moves} required to win.\\~\\
    So, to find the length of the game, we need to find minimum number of moves required to win.
\end{frame}

\begin{frame}{Rephrasing the problem}
    Each player's turn corresponds to the elimination of an arbitrary stack. In other words, if we have an array of numbers, $A$, we are removing an element from $A$ every turn.\\~\\
    If $M$ represents the number of elements that we have to remove such that \textit{all} elements in $A$ have the same value, the minimum value of $M$, $M_{\min}$ represents the minimum number of moves to win our game.
\end{frame}

\begin{frame}[fragile]{How do we find $M_{\min}$?}
    If we write $M$ as $$M=|A|-n,$$ where $n$ represents the number of common elements to keep, then it is clear that $M_{\min}$ is minimised when $n$ is maximised.\\~\\
    In other words, $M_{\min}$ is found by finding the \textit{most common element} in $A$.
\end{frame}

\begin{frame}{Finding the most common element}
    Suppose that ${\MCE}(A)$ represents the \textit{most common element} of $A$.\\~\\
    Let's take a look at how we can compute the value of $\MCE(A)$.
\end{frame}

\subsection{Side note: Key-Value Mapping Data Structure (Dictionary)}
\begin{frame}{What is a dictionary?}
    Imagine a table mapping some value $k$ to another value $v$:
    \begin{figure}
        \centering
        \begin{tabular}{c|c}
            Key ($k$) & Value ($v$)  \\\hline
            1 & 6 \\
            ``string'' & 7\\
            $\pi$ & \{1, 5\}\\
            Dog & Animal
        \end{tabular}
        \caption{A key-value table.}
    \end{figure}
    A dictionary emulates the table above; it maps an object of any type to another object of any type.
\end{frame}

\begin{frame}[fragile]{Dictionaries in Python}
    \textbf{NOTE:} The keys of a dictionary \textit{must} be distinct. Otherwise, there is an ambiguity.
    \begin{center}
    \begin{lstlisting}[language=iPython]
# Initialise an dictionary
A = {1: 2} # You can also use the dict() function...
# Add a new value to the dictionary
A[7] = 3 
# NOTE: If the key is already in the dictionary,
#       it will overwrite the value.
A[1] = 3 # e.g. this will overwrite the value at 1 to be 3. 
    \end{lstlisting}
    \end{center}
\end{frame}

\subsection{Frequency counting}
\begin{frame}{Counting frequency using a dictionary}
    We can emulate the frequency table that we manually made using a dictionary.\\~\\
    The key denotes an element of $A$, $x$, and the value denotes the frequency of $x$ in $A$, $f_A(x)$.
\end{frame}

\subsection{Summary}
\begin{frame}{The minimum number of turns is...}
    Based on our findings, we can express $M_{\min}$ as $$M_{\min}=|A|-f_A\left({\MCE}(A)\right).$$
    In other words, $M_{\min}$ is the number of elements that we need to remove so that our array solely consists of the most common element.
\end{frame}

\begin{frame}{To go first or second?}
    Whether Jill should go first or second is directly based on the parity of $M_{\min}$.\\~\\
    If Jill goes first, all \textit{odd-numbered} turns `belong' to her. As a result, if $M_{\min}$ is odd, her turn will coincide with $M_{\min}$.\\~\\
    In the same way, if Jill goes second, all \textit{even-numbered} turns `belong' to her and as a result, if $M_{\min}$ is even, her turn will coincide with $M_{\min}$.
\end{frame}

\section{``Walter's Series''}
\subsection{Overview}
\begin{frame}{Problem Description}
    Walter came up with a square series he calls the ``cool'' series. The $n$-th term of which is given by
    $$C_n=n^2-(n-1)^2.$$
    Walter wants to know the finite sum of this series: $$S_n=C_1+C_2+\ldots+C_n,$$
    and he needs \textit{your} help. For $T$ test cases, help Walter find $S_n$ modulo $10^9+7$.
\end{frame}

\begin{frame}{Two Approaches}
    There are two approaches to solving this problem. Both of them involve simplify $S_n$ (i.e. finding a closed-form expression).
\end{frame}

\begin{frame}{Method \rom{1}: Summation}
    Simplify $C_n$ gives us $C_n=2n-1$. Then, we can write $S_n$ as
    \begin{align*}
        S_n&=C_1+C_2+\ldots+C_n\\
        &=2(1)-1+2(2)-1+\ldots+2(n-1)-1+2n-1\\
        \shortintertext{We can group up the $-1$ terms that appear in each of the $n$ terms.}
        &=2(1)+2(2)+\ldots+2(n-1)+2n-n\\
        &=2(1+2+\ldots+(n-1)+n)-n\\
        &=2\left(\frac{n(n+1)}{2}\right)-n\\
        &=n(n+1)-n\\
        \Aboxed{&=n^2}
    \end{align*}
\end{frame}

\begin{frame}{Method \rom{2}: Telescoping series}
    We start by writing $S_n$ in expanded form as
    \begin{dmath}
        S_n=\left(1^2-0^2\right)+\left(2^2-1^2\right)+\left(3^2-2^2\right)+\ldots+\left((n-1)^2-(n-2)^2\right)+\left(n^2-(n-1)^2\right).
    \end{dmath}
    Notice that each successive term cancels out the one before it leaving us with $$S_n=n^2.$$
\end{frame}

\section{``Substitution''}
\subsection{Overview}
\begin{frame}{Problem Description}
    Nathan likes to play a little game called \textit{Substitutions}. From a root string $R$ and a valid string $S$, the game works by repeatedly splitting the string $S$ into two substrings $A$ and $B$ such that $A+B=S$; the new string $A+R+B$ is also considered valid.\\~\\
    Help Nathan determine whether a given string $S$ is valid for a root string $R.$
\end{frame}

\begin{frame}[allowframebreaks]{Example}
    For $R=\text{\texttt{``frA''}}$ and $S=\text{\texttt{``frfrAfrA''}},$ determine whether $S$ is valid.
    \begin{block}{Proof.}
    Assume that $S$ is valid. Then, $S$ can be made by inserting $R$ into some position of another string $X$ (i.e. $S=A+R+B$ where $A$ and $B$ are substrings of $X$). Therefore, $X$ is \texttt{``frfrA``}.
    \end{block}
    \framebreak
    \begin{block}{Proof. (contd.)}
    Likewise, $X$ is valid since $S$ was made from $X$. We repeat the process above to find another string $Y=\texttt{``fr``}$ where $X=C+R+D$ where $C$ and $D$ are substrings of $Y$.\\~\\
    Once again, since $X$ can be made from $Y$, and $X$ is valid, $Y$ must also be valid; however, $Y$ does \textit{not} contain $R$. Therefore, $Y$ is \textit{not} valid; we have reached a contradiction which means that our original string $S$ is also not valid.
    \end{block}
\end{frame}

\begin{frame}{An important observation}
    If $S$ is valid, We find that as we backtrack from $S$, every successive string ($X$, $Y$, $Z$, \ldots) will contain the root string $R$. At some point, we will reach $R$ itself.\\~\\
    If $S$ is \textit{not} valid however, we will reach a string that does \textit{not} contain $R$.\\~\\
    This property lets us determine whether $S$ is valid.
\end{frame}
\end{document}