\documentclass{concrete-book}

\usepackage{minted}

% Document metadata
\title{Course Book}
\subtitle{Algorithms and Data Structures}
\author{Shon Verch}
% \edition{Third Edition}
\dedication{Dedicated to Papa Smurf}

\begin{document}

\frontmatter
\maketitle

\tableofcontents
\cleardoublepage
\mainmatter

\part{Basic Object-Oriented Programming}

\chapter{The Principles of Object-Oriented Programming}
There is no doubt that computers are stupid. Sure, they can compute incredibly large powers and perform billions of 
computations per second but in reality, computers are nothing more than an overpowered machine which executes
incredibly primitive instructions. These instructions---called \textit{opcodes}---are most primitively exposed 
in the low-level \textit{Assembly} language. These instructions provide 
programmers the computing foundation for arithmetic, logic, and memory manipulation. However, computers are not ``smart''
in and of themselves. In order for modern computers to complete the tasks that they do (e.g. image recognition,
map directions, graphics, etc...), clever programmers and engineers need to write exact instructions for the computer.
\marginnote{Remember: the computers \textit{always} does what you tell it to. When you have a bug, it's not the 
computer, it's you.} 
For example, suppose you are making a robot and want the robot to traverse a predefined route, you would write a program which instructs
the robot to move in a fashion such that it reaches the destination:

\begin{enumerate}
    \item \textbf{Start Program}
    \item Turn left
    \item Walk straight for 10 metres
    \item Turn right
    \item Walk straight for 2 metres
    \item \textbf{End Program}
\end{enumerate}

\section{Classes}
\section{Inheritance}
\section{Polymorphism}

\chapter{Object-Oriented Python}

\part{Fundementals of Algorithms and Their Applications}

\chapter{Introduction to Algorithms}
\section{Algorithms}
\section{Algorithms in the Real-world}
\section{Analyzing Algorithms}

\chapter{Growth of Functions}
\section{Asymptotic Notation}
\section{Common Functions of Growths}

\chapter{Sorting}
\section{Insertion Sort}
\section{Quicksort}

\part{Data Structures}

\chapter{Fundemental Data Structures}
\section{Stacks}
\section{Queues}
\section{Linked Lists}

\chapter{Binary Search Trees}
\chapter{Red-black Trees}

\end{document}